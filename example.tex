% -*- tex -*-
% uspatent.tex version 1.0 9/26/2012 gmagiros@gmail.com
%
% TeX macros for writing a United States Patent Application.
% Copyright (c) 2012 George Magiros
%
% This program is free software: you can redistribute it and/or modify
% it under the terms of the GNU General Public License as published by
% the Free Software Foundation, either version 3 of the License, or
% (at your option) any later version.
%
% This program is distributed in the hope that it will be useful,
% but WITHOUT ANY WARRANTY; without even the implied warranty of
% MERCHANTABILITY or FITNESS FOR A PARTICULAR PURPOSE.  See the
% GNU General Public License for more details.
%
% You should have received a copy of the GNU General Public License
% along with this program.  If not, see <http://www.gnu.org/licenses/>.

\input uspatent.tex

\timestamp	
\docketnumber{24207-12761}

\title{CONTROLLING COMMUNICATION WITHIN A CONTAINER DOCUMENT}

% Patent No.: US 7,958,516 B2
% Appl. No.: 11/737,124
% Assignee: Google, Inc

\twoinventors{Michael Buerge, Seilerstrasse 13 (CH)}
             {Bernhard Seefeld, Wuhrstr. 32 (CH)}

\crossreference

This application is related to co-pending
U.S. application Ser. No. 11/456,703, filed Jul. 11, 2006, which is
incorporated by reference in its entirety.

\background

This invention relates generally to controlling
communication within a container document.

Many browsers prevent code that is hosted on one domain
from accessing code that is hosted on a different domain. Where a
document is hosted on one domain and references content hosted on a
different domain, the document is typically precluded from
communicating with or accessing the content on the different
domain. Such a situation may arise on a portal web page, for example,
which is designed to allow a user to select a restaurant from a frame
hosted by a city guide domain and then cause directions to the
selected restaurant to be generated in a frame that is hosted by a
mapping domain. Direct communication between the frames hosted by the
different domains is typically disallowed, due in part to security
restrictions and other limitations of the existing browser
programs.

\summary

To allow communication
between modules associated with different domains (such as when
presented within a container document), various embodiments of the
present invention provide mechanisms to work around the limitations of
existing browser programs. A system allows modules associated with
different domains to communicate, such as within a container
document. To transfer payload data from the first module associated
with a first domain to a second module associated with a different
domain, the first module adds the payload data as a text string to the
URL of a transport module associated with the second domain. This way,
the second module may directly access the modified transport module to
obtain the payload data from its URL. The second module may likewise
add other payload data as a text string to the URL of another
transport module associated with the first domain, thereby enabling
communication from the second domain to the first.

In one embodiment, the first and second modules create
inline frames, or IFRAMES, contained within a container document,
which may implement a web portal. Each transport frame may also be an
empty IFRAME that is hidden in the container document.

In another embodiment, the second module polls a transport
module to determine whether data has been transferred to the second
module. This polling may be periodic or otherwise determined according
to an algorithm. In this sense, the data transport mechanism may be
asynchronous.

Beneficially, communicating the
payload data via the URL of a transport module may allow the
cross-domain communication without requiring communication between the
corresponding servers, as the communication may occur entirely at the
client machine on which the browser program and modules are
executed.

\drawings

FIG. 1 is a block diagram of an overall container document
system architecture for providing content to a user via a container
document, in accordance with an embodiment of the invention.

FIG. 2 depicts an illustrative container
document, in accordance with an embodiment of the invention.

FIG. 3 illustrates a container document that
includes functional modules and transport modules, in accordance with
an embodiment of the invention.

FIG. 4 is an
interaction diagram of a process for communication within the
container document of FIG. 3 between modules associated with different
domains, in accordance with an embodiment of the invention.

The figures depict various embodiments of the
present invention for purposes of illustration only. One skilled in
the art will readily recognize from the following discussion that
alternative embodiments of the structures and methods illustrated
herein may be employed without departing from the principles of the
invention described herein.

\description

Exemplary embodiments are discussed in detail below. While
specific exemplary embodiments are discussed, it should be understood
that this is done for illustration purposes only. A person skilled in
the relevant art will recognize that other components and
configuration can be used without departing from the spirit and scope
of the claimed inventions. Various embodiments of the present
invention relate to controlling communications within a container
document.

Illustrative Container Document Environment for Use of Embodiments

One illustrative example of a
container document may be such as one used in connection with a
personalized portal size. A personalized portal site (e.g., My Yahoo!,
start.com, or Google Personalized Homepage) may allow the user to
select only content (e.g., interactive, read-only, updating, data
feeds, etc.) to display on a personalized page, such as a new email
alerts, current weather and/or traffic conditions, movie show times,
horoscopes, etc. According to one embodiment of the present invention,
the illustrative example of using modules may involve modules that may
be incorporated into a personalized portal page (one example of a
container document) along with modules developed (e.g., by an a third
party developer) for inclusion in the container.

Therefore, for illustrative purposes, an explanation of the
container documents or modules is provided here, but it should be
appreciated that the various embodiments may be employed in connection
within other contexts as well. In addition, details regarding the
modules are provided in detail in four co-pending and commonly
assigned patent applications, all of which are hereby incorporated by
reference in their entirety. Such application are: U.S. patent
application Ser. No. 11/298,930, filed Dec. 12, 2005, and entitled
``Remote Module Incorporation into a Container Document'';
U.S. patent application Ser. No. 11/298,922, filed Dec. 12, 2005, and
entitled ``Module Specification for a Module to be Incorporated
into a Container Document''; U.S. patent application
Ser. No. 11/298,987, filed Dec. 12, 2005, and entitled
``Customized Container Document Modules Using
Preferences''; and U.S. patent application Ser. No. 11/298,988,
filed Dec. 12, 2005, and entitled ``Proxy Server Data
Collection.'' While details are provided in this application, in
general, such a system may be understood by the following.


The system may comprise a number of
components. The system may comprise a container server that serves a
container document (e.g., a personalized page). The container document
``contains'' one or more modules, including one or more
remote modules. As used herein, the term ``container
document'' or ``container'' should be understood to
include a personalized homepage of a website, a sidebar, toolbar
element that incorporates one or more such modules, a page hosted by a
site, a document capable of rendering modules (e.g., any document
capable of rendering HTML code or XML code) in the format of the
module (e.g., XML). Also, the container may be a website of another
entity that incorporates the modules when supplied the modules through
a syndication system.


As used herein, the
term ``module'' may be understood to refer to a piece of
software and/or hardware that renders data for use in a container
document. Modules may be personalized to user preferences, preferences
of the container, preferences of the environment or other inputs. In
an exemplary embodiment, various modules may or may not be visible to
the viewer of the container document. A module specification may be
understood to include a set of instructions used to render data for
the container document using elements that have been
predefined.

Overview and System Architecture of Container Document
System


\ref{hostserversystem}{host server system 10}
\ref{personalizedcontainerserver}{personalized container server 10}
\ref{containerserver}{container server 12}
\ref{Containerserver}{Container server 12}
\ref{moduleserver}{module server 14}
\ref{Moduleserver}{Module server 14}
\ref{specificationserver}{specification server 16}
\ref{Specificationserver}{Specification server 16}
\ref{backendserver}{back end server 18}
\ref{Backendserver}{Back end server 18}

\ref{databasesystems}{database systems 20}
\ref{usersystem}{user system 22}
\ref{usersystems}{user systems 22}
\ref{remotesourcesystems}{remote source systems 24}
\ref{remotesourcesystem}{remote source system 24}
\ref{network}{network 26}
\ref{analysismodule}{analysis module 28}

\ref{developersystem}{developer system 30}
\ref{developersystems}{developer systems 30}
\ref{modulecreationservers}{module creation servers 32}
\ref{modulecreationserver}{module creation server 32}
\ref{Modulecreationserver}{Module creation server 32}
\ref{syndicationserver}{syndication server 34}
\ref{Syndicationserver}{Syndication server 34}
\ref{advertisementserver}{advertisement server 36}
\ref{adserver}{ad server 36}
\ref{Adserver}{Ad server 36}
\ref{geocodeserver}{geocode server 37}
\ref{recipientservers}{recipient servers 38}
\ref{recipientserver}{recipient server 38}
\ref{mapserver}{map server 39}



FIG. 1 depicts an overall system
diagram in which the communication embodiments may be employed
according to one embodiment of the present invention. As illustrated,
FIG. 1 may comprise a \hostserversystem{} with a plurality of
modules that may be associated therewith. Such modules may comprise a
\containerserver{}, a \moduleserver{}, a \specificationserver{}, a 
\backendserver{}, an \analysismodule, a \modulecreationserver{}, a \syndicationserver, 
an \advertisementserver, a \geocodeserver{} and a \mapserver{}. As illustrated, \personalizedcontainerserver{} may connect over
a \network{} to a plurality of systems.



Other systems connected
to the network may comprise one or more \usersystems{}, one or
more \remotesourcesystems, one or more module \developersystems{} and one or more syndication \recipientservers{}. In addition, one or more \databasesystems{} may
operate in conjunction with the various modules of \hostserversystem{}.



\Containerserver{} may
serve the container document to \usersystems{} over \network. \Containerserver{} may comprise a web server or
related server systems that takes data and/or instructions and
formulates a container for transmission over the network to the \usersystem. 
It should be appreciated, however, that 
\containerserver{} may reside on \usersystem{} as well so that a
network connection may not be used. In the example in which the
container document comprises a word processing document, for example,
\containerserver{} may comprise a word processing module.


\Moduleserver{} may provide data from
modules to \containerserver{} for incorporation into a
container document. It should be appreciated that in one embodiment,
\containerserver{} and \moduleserver{} may comprise a
single unit performing both functions. \moduleserver{} may
provide data for the container document by interpreting and/or parsing
instructions in the module specification associated with the
module. According to one embodiment, \moduleserver{} may serve
the module content to the container document through the use of a
browser inline frame (IFRAME). An IFRAME may be generally understood
to be an independently operated browser window instance inside the
container document. One advantage of an IFRAME is that is protects the
container document from the IFRAME's content and vice versa, e.g.,
JavaScript on the container document may not be permitted to access
any JavaScript code in the inner IFRAME (same for CSS, DOM, or cookie
objects). In an exemplary embodiment, this failure to permit access to
any JavaScript code in the inner IFRAME may be the result of security
settings of the browser.


To expedite display
of container documents, modules may be displayed inline within the
container document. Inline display may be understood as referring to
display with other document elements. One example is a display
generated from code for HTML in the body according to HTML
standards. In one embodiment, \moduleserver{} or some other
component may determine whether the module is deemed trusted before
including it in the container document inline due to the risks of
various security issues an inline module could create. According to
one embodiment, the module may incorporate an indicia of approval
(e.g., digital certificate) issued by the container module or an
entity associated with the container module as described in detail
below. If the indicial of approval is present, \moduleserver{}
may render the data from a module for inline presentation in the
container document.


\Specificationserver{}
provides the module specification file to \moduleserver{}. The module specification may be cached and stored in a
database accessible to the \moduleserver{} and/or
\specificationserver{} or may be retrieved from a location
associated with the specification as detailed later. For example,
\specificationserver{} may reside on a \remotesourcesystem. In addition, \specificationserver{} may be
connected to module server over a network with the module
specification located at another location on the network accessible to
\specificationserver{}.


\Backendserver{} may be provided for interacting with one or more
databases (e.g., large or dynamic databases of information). For
example, for a news module that obtains frequent updates and demands a
flow of data, (e.g., from an RSS feed), \backendserver{} may
format the data into HTML for the container.



In one specific example, a person may create a module
(e.g., a maps module), such as one that uses an application program
interface (API) to an existing mapping program to create a module to
display a map of downtown Mountain View, Calif. The module may
comprise an XML specification file or module specification file stored
on a specification server. The specification server may comprise any
server, including one on the site from which the container page is
hosted or any other site. The user or another person may then include
this new module on a personalized homepage (container document). The
server that serves the container document may operate as the module
server and the server that generates the mapping data through an
inquiry from its API may be considered to be the backend server.



According to one embodiment of the present
invention, \analysismodule{} may analyze modules at various
times (e.g., when the module is first selected by a user, each time
the module is called by a container for inclusion or at any other time
determined to be advantageous for safety and security and other
times). \analysismodule{} may perform a number of actions,
including comparing the module with a list of disapproved or dangerous
modules or a list of approved modules. The comparison may involve
exact or substring (e.g., prefixes, suffixes and regular expressions)
matching by name or location (e.g., URL), contents of the
specification, contents of the location where the specification
resides, or information that may be ascertainable about the
module. \analysismodule{} may take one or more actions in
response to a determination that the module is disapproved or
dangerous, including, for example, silently blocking the request,
(i.e., providing a generic error), blocking the request with an error
that explains the reason it was blocked or redirecting the request to
a different module specification that has been determined to be safe
and related to the disapproved module (e.g., another module that
relates to maps, if the first one was a disapproved mapping site or a
module that includes the keyword ``basketball'' if the
disapproved module was a basketball module). For example, through
redirection, the URL of the original module may be passed to the
``safe'' module. The safe module may then use a proxy
server, as described below, to retrieve the original URL's
content. Developers may then replace the error handler to fix small
bugs in the original module to be able to display the content of the
original module. In another embodiment, \analysismodule{} may
parse the module content to determine whether it is safe, such as by
compiling JavaScript or other scripts contained in the module to try
to identify unsafe or undesired actions the module may perform.



One or more \modulecreationservers{}
may be provided. This server may operate as a ``wizard''
to enable module creators to create a module through an interactive
process controlled by \modulecreationserver{}. For example,
\modulecreationserver{} may provide a series of user
interfaces that enable the module creator to provide inputs that are
then used by the module creator to automatically generate a module
specification. For example, various module specification templates may
be provided with corresponding inputs. \Modulecreationserver{} may
then take inputs supplied by a module creator, insert them into the
template and then generate the module specification for the module. A
preview, testing and debugging function may also be offered as part of
this ``wizard.'' This module may be downloadable as well so it may be
installed and operated at any node on the network.



A \syndicationserver{} may prepare data for transmission to one or more syndication
\recipientservers{} related to modules. \Syndicationserver{} may receive a request for a module and/or module content and
deliver that module or content to a syndication \recipientserver{} over \network{}. \syndicationserver{} may
reside at \hostserversystem{} or at another location on the
network. For example, if an operator of a sports web site (an example
of a syndication recipient system 38) desired to include a maps
module created by a \remotesourcesystem, it may do so
through a request to \syndicationserver{}. \Syndicationserver{} may then cooperate with \moduleserver{} to generate
data for the container document (here the sports web site page of the
syndication recipient system 38). That may involve retrieving
the module specification from \remotesourcesystem,
supplying preferences received from the syndication \recipientserver{} (e.g., city information for the sports team of a page being
displayed) and/or generating data for the container. It is also
possible that the data may be rendered at syndication \recipientserver{} into its container document in either an IFRAME or
inline. \syndicationserver{} may thus syndicate modules
accessible to it. It may do so based on requests for specific modules
or other criteria it determines (e.g., content matches, keyword
matches, monetary values associated with modules and/or syndication
requesters, etc.)



\Adserver{} may
provide advertisements associated with modules to containers. For
example, an advertisement may be incorporated with module data when
data is delivered to a container document. \Adserver{} may
operate with \syndicationserver{} to deliver advertisements to
syndication \recipientservers{} based on a syndication request
for a module. The advertisements may be selected by \adserver{} based on a wide variety of criteria, including, but not
limited to, the relationship between the content of or information
about the container, module, other modules in the container,
syndication \recipientserver{} information, monetary
elements/relationships related to any of the foregoing and/or
combinations thereof. \Adserver{} may comprise the Google
AdSense system, according to one embodiment of the present
invention. \Adserver{} may operate as described in one or more
of the following patent applications, the subject matter of which is
hereby incorporated by reference in their entirety. Specifically, \adserver{} may manage online advertising by associating two or
more concepts related to a module with an advertisement and
associating a bid, collectively, with the two or more keywords in the
manner discussed in the context of serving advertisements with
electronic documents in U.S. patent application Ser. No. 10/340,193,
filed on Jan. 10, 2003, entitled ``Pricing Across Keywords
Associated with One or More Advertisements,'' which is
incorporated by reference herein in its entirety. Additional examples
of presenting advertisements and managing advertising costs are
discussed in U.S. patent application Ser. No. 10/340,543, filed on
Jan. 10, 2003, entitled ``Automated Price Maintenance for Use
With a System in which Advertisements are Rendered with Relative
Preferences,'' and U.S. patent application Ser. No. 10/340,542,
filed Jan. 10, 2003, entitled ``Automated Price Maintenance for
Use With a System in which Advertisements are Rendered with Relative
Preference Based on Performance Information and Price
Information,'' which are incorporated by reference herein in
their entirety.



A \geocodeserver{}
may be provided to generate geocode information from location
descriptions as is known in the art. A \geocodeserver{} may
generate latitude and longitude numeric values from geographic
locations.



A \mapserver{} may
generate map output. Mapping systems, such as Google Maps and Google
Earth, may be used to generate this data.



One
or more \databasesystems{} may be provided that store, in any
number of ways, container information, module specifications and/or
related information, formatting data, per-user and per-module
preference data, remote module ID data, remote module location
reference data, advertisement data, advertiser data, content/monetary
data, syndication recipient data, templates for modules, inputs for
modules, lists of trusted and untrusted modules, approval criteria and
related information and/or any other information used by the modules
to operate as described herein. While a single database structure is
shown, it is well understood that the data may be stored at a number
of locations and in one or more systems.



While one configuration is shown in FIG. 1, it should be
appreciated by one of ordinary skill in the art that other
configurations of these various modules may also be possible. For
example, the various modules depicted within \hostserversystem{}
may be disposed at various locations around \network{} or at various points on several networks. In addition,
whereas a single \hostserversystem{} is depicted, it should
be appreciated that any number of each of the modules depicted on
FIG. 1 may be provided including \network{}.



In one embodiment, \network{} may comprise the
Internet. Other networks may also be used for connecting each of the
various systems and/or servers.



In addition,
what is shown as \usersystem{} may also operate as a \remotesourcesystem{} and/or a module \developersystem. In
other words, one computer system may operate in different capacities:
as a user system, as a remote source system, as a syndication server,
as a target content server, and/or a module \developersystem. In
addition, as explained in greater detail below, each of the modules
depicted within \hostserversystem{} may also be disposed at a
\usersystem{}, a \remotesourcesystem, or a module
\developersystem. Similarly, \databasesystems{} may be
associated with each of the modules depicted within FIG. 1 depending
upon the configuration desired.

Container Document Including Modules


According to one embodiment of the present
invention, systems and method are provided to incorporate modules into
a container document. One example of a container document would be a
personalized home page, such as the Google Personalized Homepage
currently available to users of the Google services on the
Internet. Instead of restricting the types of content that a user is
able to include in a container document such as a personalized home
page, one or more embodiments of the present invention enable users to
select modules from sources other than the source of the container
document. So, for example, a user may elect to include a module in his
or her personalized Google home page from a source not associated with
Google.



It should be appreciated that various
forms of the container document may exist, but one such illustrative
example is depicted in FIG. 2. FIG. 2 depicts a container page
100 with a plurality of modules that have been incorporated
into the container document. A plurality of methods of incorporation
are possible, including the use of the IFRAME and inline HTML
techniques. These issues will be discussed in greater detail
below. FIG. 2 depicts a plurality of modules including a photo remote
module 101, a task list module 102, a game module
104, a stock module 105, a maps module 106, a
remote module 108, a remote module 210, a remote module
312, and a remote module 114. Different formats for the
various modules may exist depending upon the specifications provided
by the creator of the module. As depicted, some modules may provide
scroll bars, and others may not. Some modules may be different sizes
or shapes than other modules. In addition, some modules may offer the
opportunity for the user to edit the display preferences and/or
per-use data associated with the module. (See, for example, modules
102, 104, 105, 106 and 110 that
provide an edit link.) For example, if the module relates to a maps
module 106, the user may be provided the opportunity to edit an
address or addresses that are mapped in that module. In one
embodiment, inlined modules may be automatically sized by a container
document so no scrolling, height or scaling information may be
provided. If a module developer wants the module to have these
properties in this embodiment, an inlined module may be wrapped with a
fixed size $<$DIV$>$ tag and content placed in the tag. The
scroll bar, height and other characteristics may be then specified for
the inlined content. One of the attributes allows specifying
scaling=``...''to let the developer indicate how a
module may be scaled up or down for different sizes of placements in
the container documents.



One of the functions
provided with this example container document 100 is the
opportunity to add content to the container page through selecting the
add content element 103. Upon selecting ``add
content'' element 103, the system may offer the user the
opportunity to interactively indicate an additional element to be
included in the container page. Various forms of an interface to
receive that input may be provided. One illustrative form is presented
in FIG. 2 toward the bottom of the page in section 120. In that
section, the user may be presented with an interface element to select
from a browsable list of modules that may be arranged into a
categorization structure. Another section of input section 120
may enable the user to specify a reference to a location for a module
to be incorporated into the container document. Such a section may be
such as that depicted through an input box 126 with a submit
element 128. In one illustrative example, the user may specify
a location reference (e.g., the uniform resource locator (URL)) where
data exists related to a module to be incorporated. As explained in
greater detail below, one example of the data is an XML-based file
that meets the scripting preferences of the operator of the container
document system 10.



At its base level,
the specification may comprise a plurality of elements including the
XML version information, module preferences, which may be optional,
user preferences, which may be optional, a content type designator and
then a content element that is used to populate the portion of the
container allocated for the module. It should be appreciated that the
content may be specified in various forms of code, typically code that
is interpretable by a user system when generating the container for
presentation. Such code may include HTML, JavaScript, or other forms
of code that may be used to depict the format of a web page.



According to another embodiment of the present
invention, the module specification may be embedded in one or more
other files or computer instructions. According to this embodiment,
the \moduleserver{} may, when provided with the identification
of data for generating a module, look for a module specification
within the data.Transport Mechanism for Communicating Between
Modules



According to exemplary embodiments,
IFRAMEs may be used to control communication within a container
document. As noted above, module content may be presented in an IFRAME
hosted on a domain separate from the domain of the container
server. In such an embodiment, policies may exist that prevent an
IFRAME that may be hosted on the separate domain from communicating
with and/or accessing the container server. One such policy, for
example, may be referred to as the ``Same Origin
Policy.''



In various systems,
communication between modules and/or the container document may be
controlled. For example, this communication may be controlled to limit
access to the domain of the particular module or containing window
that is trying to communicate.



As used
herein, domain should be understood to be a source, such as a single
DNS entry (e.g., www.google.com) or related DNS entries (e.g., all
registrations with the base name www.google.com) or sometimes referred
to as a trust domain. In addition, as used herein, the term module may
refer to the container document itself or to another module that
provides content to the user from another domain. Accordingly,
communications can be controlled between modules within a container
document and associated with different domains, or between the
container document itself and a module associated with a different
domain than the container document.



A module
may reference a frame, which may be presented in a container
document. In an exemplary embodiment, the container document and one
or more other modules may each include code, for example, that
references an IFRAME. A module may also include code for receiving a
parameter. For example, the container document may include a script
defining a ``ProcessPayload'' function that receives
``payload'' as a parameter. In such an embodiment, the
``ProcessPayload'' function may be any function that
receives the ``payload'' and processes the
``payload,'' where the ``payload'' represents
a structure of a string of characters that can represent any
value. While the ``ProcessPayload'' function may be
included in the container document and any other module, it will be
understood that the ``ProcessPayload'' function does not
have to be defined within any particular module, and may be defined
and/or executed elsewhere. Further, in an exemplary embodiment, the
logic encapsulated by such a function may be executed separate from
the function.



One or more of the modules in
the container document may include a script (e.g., var
iframe\_1=document.createElement(``IFRAME'')) to
construct an IFRAME in the container document. The constructed IFRAME
may include module content for displaying the module within the
container document. In one embodiment, the modules construct one or
more IFRAMEs upon the loading of the container document; however, and
IFRAME may also be dynamically constructed.



FIG. 3 illustrates an example system in which modules from
different domains can communicate within a container document. In this
example, the container document creates IFRAME\_A 315, which is
associated with domain a.com, and IFRAME\_B 325, which is
associated with domain b.com. The container document then loads
Module\_A 310 into IFRAME\_A 315 and Module\_B 320
into IFRAME\_B 325. In this way, the IFRAMEs 315 and
325 can be used for displaying and enabling user interaction
with content from their associated domains. This avoids having foreign
code (i.e., code from another domain) run in the scope of the
container document, but rather encapsulated in an IFRAME.



The container document may further include a
first transport module 330 and a second transport module
340, which may also be created upon loading of the container
document, or they can be created dynamically. In one embodiment, the
transport modules 330 and 340 are implemented hidden
IFRAMEs that have no content or code therein.



With reference to the example system of FIG. 3, FIG. 4
illustrates a process for enabling communication within the container
document between a module hosted on one domain (e.g., Module\_A) and a
module hosted on a different domain (e.g., Module\_B). In this example,
the first transport module 330 is associated with domain a.com,
and the second transport module 340 is associated with domain
b.com. In this process, Module\_A 310 has generated 410
or otherwise obtained information to be communicated to Module\_B
320. As explained, Module\_A 310 cannot directly
communicate data to Module\_B 330 because they are associated
with different domains.



In one embodiment,
Module\_A 310 incorporates the data to be communicated as a text
string, and this payload data is added 420 to the URL reference
of the second transport module 340. As mentioned, any module
may be referenced by a URL reference, which may incorporate variable
information. In one embodiment, the URL reference may include a
hashing symbol ``\#'' and append variable information
(``payload'') to the URL. This portion of the URL
following the hashing symbol may be referred to as the fragment ID or
the anchor of the URL. While a hashing symbol is shown and described,
any other mechanism to incorporate the variable information and/or
parameters into the URL may be used, such as, for example, through a
filename, directory path, or subdomain name in the URL. Accordingly,
in one embodiment, Module\_A 310 communicates the payload by
setting 420 the URL reference of the second transport module
340. For example, Module\_A 310 may prepare the payload
data and embed it in the URL associated with the second transport
module 340 in the form: http://b.com/path\#$<$payload$>$.



The size of the payload may be
limited by a particular browser program being used. For example, a
typical browser may limit the URL to 4 kilobytes, so the payload is
limited by that much less the size of the beginning portion of the
URL. Even without an externally enforced limit, an embodiment of the
invention may limit the size of the payload (or URL). For example, too
large a payload (e.g., greater than 10 kilobytes) may result in high
latency. The setting of these parameters will depend on the
application of the communication techniques described herein.



In a typical embodiment, Module\_B 320
will not necessarily know when the second transport module 340
has payload data that Module\_B 320 can read. Accordingly, in
one embodiment, Module\_B 320 polls 440 the second
transport module 340 to check if it has any new payload
data. As illustrated in FIG. 4, Module\_B 320 may poll
440 the second transport module 340 periodically, and
this polling may be at regular intervals. Alternatively, the frequency
of polling can be adjusted dynamically based on how recent the last
payload arrived. For example, the polling may be more frequent
following a transfer of payload data, as it may be considered more
likely that additional data will be sent following the arrival of
other data. It can be appreciated that the maximum polling interval
can be set or otherwise constrained by the allowable latency in the
communication, while too frequent polling can undesirably drain
computing resources.



Once Module\_B 320
polls 440 the second transport module 340 and determines
that it has obtainable payload data, Module\_B 320 reads
450 the payload data from the URL reference of the second
transport module 340. In this way, communication of this
payload data from a module of domain a.com to a module of domain b.com
has been achieved.



In one embodiment, to
enable further communication from Module\_A 310 to Module\_B
320, the process described above may be repeated. In addition,
Module\_B 320 may communicate data to Module\_A 310 in the
reverse direction, for example, by setting the URL reference of the
first transport module 330. Such a communication may be used
for substantive data transfer or just to send an acknowledgement
message that the payload has been received. To enable it to receive
data, Module\_A 310 may poll 470 the first transport
module 330 to check whether it has been loaded with new data.



By repeating the process described above,
Module\_A 310 and Module\_B 320 can communicate in either
direction. In one embodiment, where multiple payloads are
communicated, the modules track the messages by labeling each payload
with a sequence number. Additionally, the container document may
contain modules from other domains as well. In such a case, there may
be one or more transport modules initially associated with those other
domains, allowing the modules from the other domains to communicate as
well.


Alternative Environments



It is noted
that the processes and systems described herein may be used in other
contexts and environments within the scope of the invention. For
example, the container document may comprise a page generated from a
hosted page creator (e.g., geocities.com or pages.google.com,
etc.). In this context, an IFRAME in a page being created or
previously created may include a module (e.g., a plugin) that is
maintained on a domain different from the hosted page creator domain
and the module may then utilize an IFRAME to generate content that
permits modification of the page being created for the hosted page
creator may reference include an IFRAME that permits modification of
the page being created or already created in a safe manner.



Further, an IFRAME that wraps the code also may
be invisible to the user on the container document (e.g., webpage). As
a result, using this technique, user interaction may be eliminated, in
one exemplary iteration, to employ the IFRAME in an IFRAME technique
described herein.



In another example, any web
page (e.g., personal, corporate, educational, etc.) may use the IFRAME
within an IFRAME method and system to include features or content from
a third party while limiting the third party's ability to detect
cookies on the server of the web page or create other security/privacy
problems.



The foregoing description of the
embodiments of the invention has been presented for the purpose of
illustration; it is not intended to be exhaustive or to limit the
invention to the precise forms disclosed. Persons skilled in the
relevant art can appreciate that many modifications and variations are
possible in light of the above teachings.



Some portions of above description describe the embodiments
of the invention in terms of algorithms and symbolic representations
of operations on information. These algorithmic descriptions and
representations are commonly used by those skilled in the data
processing arts to convey the substance of their work effectively to
others skilled in the art. These operations, while described
functionally, computationally, or logically, are understood to be
implemented by computer programs or equivalent electrical circuits,
microcode, or the like. Furthermore, it has also proven convenient at
times, to refer to these arrangements of operations as modules,
without loss of generality. The described operations and their
associated modules may be embodied in software, firmware, hardware, or
any combinations thereof.



Embodiments of the
invention may also relate to an apparatus for performing the
operations herein. This apparatus may be specially constructed for the
required purposes, or it may comprise a general-purpose computing
device selectively activated or reconfigured by a computer program
stored in the computer. Such a computer program may be stored in a
computer readable storage medium, such as, but not limited to, any
type of disk including floppy disks, optical disks, CD-ROMs,
magnetic-optical disks, read-only memories (ROMs), random access
memories (RAMs), EPROMs, EEPROMs, magnetic or optical cards,
application specific integrated circuits (ASICs), or any type of media
suitable for storing electronic instructions, and each coupled to a
computer system bus. Furthermore, the computers referred to in the
specification may include a single processor or may be architectures
employing multiple processor designs for increased computing
capability.



Embodiments of the invention may
also relate to a computer data signal embodied in a carrier wave,
where the computer data signal includes any embodiment of a computer
program product or other data combination described herein. The
computer data signal is a product that is presented in a tangible
medium and modulated or otherwise encoded in a carrier wave
transmitted according to any suitable transmission method.



The algorithms and displays presented herein
are not inherently related to any particular computer or other
apparatus. Various general-purpose systems may also be used with
programs in accordance with the teachings herein, or it may prove
convenient to construct more specialized apparatus to perform the
required method steps. The required structure for a variety of these
systems will appear from the description above. In addition,
embodiments of the invention are not described with reference to any
particular programming language. It is appreciated that a variety of
programming languages may be used to implement various embodiments of
the invention as described herein, and any references to specific
languages are provided for disclosure of enablement and best mode of
embodiments of the invention.



Finally, it
should be noted that the language used in the specification has been
principally selected for readability and instructional purposes, and
it may not have been selected to delineate or circumscribe the
inventive subject matter. Accordingly, the disclosure of the
embodiments of the invention is intended to be illustrative, but not
limiting, of the scope of the invention, which is set forth in the
following claims.  

\claims


A method for facilitating communication between modules associated with different domains in a browser application, the method comprising:
\el creating a transport module associated with a second domain, the transport module having a URL reference;
\el adding payload data from a first module associated with the first domain to a portion of the URL reference of the transport module; and
\el reading the payload data by a second module associated with the second domain.


The method of claim 1, wherein the first and second modules comprise inline frames contained within a container document.


The method of claim 1, wherein the transport module comprises an empty inline frame.


The method of claim 1, wherein the payload data is added to a fragment ID following a hash symbol in the URL of the transport module.


The method of claim 1, further comprising:
\el adding payload data from the second module to a portion of the URL reference of a transport module associated with the first domain; and
\el reading the payload data in the transport module associated with the first domain by the first module.


\label{pollingclaim} The method of claim 1, further comprising:
\el periodically polling the transport module by the second module, the polling at least in part checking the domain associated with the transport module.


The method of \pollingclaim{}, wherein the frequency of the polling is dynamically adjusted based at least in part on a time since a previous communication with the first module.


A computer program product for facilitating communication between modules associated with different domains in a browser application, the computer program product comprising a computer-readable medium containing computer program code for performing the method comprising:
\el creating a transport module associated with a second domain, the transport module having a URL reference;
\el adding payload data from a first module associated with the first domain to a portion of the URL reference of the transport module; and
\el reading the payload data by a second module associated with the second domain.


The computer program product of claim 1, wherein the first and second modules comprise inline frames contained within a container document.


The computer program product of claim 1, wherein the transport module comprises an empty inline frame.


The computer program product of claim 1, wherein the payload data is added to a fragment ID following a hash symbol in the URL of the transport module.


The computer program product of claim 1, further containing computer program code for:
\el adding payload data from the second module to a portion of the URL reference of a transport module associated with the first domain; and
\el reading the payload data in the transport module associated with the first domain by the first module.


\label{checkingclaim} The computer program product of claim 1, further containing computer program code for:
\el periodically polling the transport module by the second module, the polling at least in part checking the domain associated with the transport module.


The computer program product of \checkingclaim{}, wherein the frequency of the polling is dynamically adjusted based at least in part on a time since a previous communication with the first module.

\abstract

 A system allows modules associated with different domains to communicate, such as within a container document. To transfer payload data from the first module associated with a first domain to a second module associated with a different domain, the first module adds the payload data as a text string to the URL of a transport module associated with the second domain. This way, the second module may directly access the modified transport module to obtain the payload data from its URL. The second module may likewise add other payload data as a text string to the URL of another transport module associated with the first domain, thereby enabling communication from the second domain to the first.


\bye
