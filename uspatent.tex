% -*- tex -*-
% uspatent.tex version 1.0 9/26/2012 gmagiros@gmail.com
%
% TeX macros for writing a United States Patent Application.
% Copyright (c) 2012 George Magiros
%
% This program is free software: you can redistribute it and/or modify
% it under the terms of the GNU General Public License as published by
% the Free Software Foundation, either version 3 of the License, or
% (at your option) any later version.
%
% This program is distributed in the hope that it will be useful,
% but WITHOUT ANY WARRANTY; without even the implied warranty of
% MERCHANTABILITY or FITNESS FOR A PARTICULAR PURPOSE.  See the
% GNU General Public License for more details.
%
% You should have received a copy of the GNU General Public License
% along with this program.  If not, see <http://www.gnu.org/licenses/>.


%%%%%%%%%%%%%%%%%%%%%%%%%%%%%%%%%%%%%%%%%%%%%%%%%%%%%%%%%%%%%%
% macros for setting the font size to 12 points
%%%%%%%%%%%%%%%%%%%%%%%%%%%%%%%%%%%%%%%%%%%%%%%%%%%%%%%%%%%%%%

% roman 
\font\twelverm   =cmr12    % twelve point roman
\font\ninerm     =cmr9     % nine point roman
\font\sevenrm    =cmr7     % seven point roman

% math italics
\font\twelvemi   =cmmi12   % twelve point math italics
\font\ninemi     =cmmi9    % nine point math italics
\font\sevenmi    =cmmi7    % nine point math italics

% math extensions
\font\twelveex   =cmex10 at 12pt  % twelve point math extension
\font\nineex     =cmex9    % nine point math extension
\font\sevenex    =cmex7    % seven point math extension

% math symbols
\font\twelvesy   =cmsy10 at 12pt  % twelve point math symbol
\font\ninesy     =cmsy9    % nine point math symbol
\font\sevensy    =cmsy7    % nine point math symbol

% italics, slant, typewriter
\font\twelveit   =cmti12   % twelve point italics
\font\twelvesl   =cmsl12   % twelve point slant
\font\twelvett   =cmtt12   % twelve point typewriter text

% bold
\font\twelvebf   =cmbx12   % twelve point bold
\font\ninebf     =cmbx9    % nine point bold
\font\sevenbf    =cmbx7    % seven point bold

% set font size to 12 points
\def\twelvepoints{
  \def\rm{\fam0\twelverm}
  \normalbaselineskip=14.4pt
  \textfont0=\twelverm \scriptfont0=\ninerm \scriptscriptfont0=\sevenrm
  \textfont1=\twelvemi \scriptfont1=\ninemi \scriptscriptfont1=\sevenmi
  \textfont2=\twelvesy \scriptfont2=\ninesy \scriptscriptfont2=\sevensy
  \textfont3=\twelveex \scriptfont3=\nineex \scriptscriptfont3=\sevenex
  \textfont\itfam=\twelveit  \def\it{\fam\itfam\twelveit}
  \textfont\slfam=\twelvesl  \def\sl{\fam\slfam\twelvesl}
  \textfont\ttfam=\twelvett  \def\tt{\fam\ttfam\twelvett}
  \textfont\bffam=\twelvebf  \def\bf{\fam\bffam\twelvebf}
  \scriptfont\bffam=\ninebf  \scriptscriptfont\bffam=\sevenbf
}


%%%%%%%%%%%%%%%%%%%%%%%%%%%%%%%%%%%%%%%%%%%%%%%%%%%%%%%%%%%%%%
% private utility macros
%%%%%%%%%%%%%%%%%%%%%%%%%%%%%%%%%%%%%%%%%%%%%%%%%%%%%%%%%%%%%%

% dynamically creates a macro
\def\makedef#1#2{%
  \expandafter\xdef
  \csname #1%
  \endcsname{#2}%
}

% turns on centering
\def\center{
  \everypar={}
  \parindent 0pt
  \rightskip 0pt plus 1fill
  \leftskip 0pt plus 1fill
  \hyphenpenalty=200
}

% centers text for a heading
\def\centered#1{{                       
  \center
  #1
  \par
  \vskip\normalbaselineskip
}}


%%%%%%%%%%%%%%%%%%%%%%%%%%%%%%%%%%%%%%%%%%%%%%%%%%%%%%%%%%%%%%
% page numbering macros
%%%%%%%%%%%%%%%%%%%%%%%%%%%%%%%%%%%%%%%%%%%%%%%%%%%%%%%%%%%%%%

% set the page numbers in the footline
% ------------------------------------
% 37 CFR 1.52 (b) (5) Other than in a reissue application or reexamination 
% proceeding, the pages of the specification including claims and abstract 
% must be numbered consecutively, starting with 1, the numbers being 
% centrally located above or preferably below, the text. 
\footline={%
  \vbox{%
    \rm\everypar={}\parindent 0pt%
    \vskip 45pt%
    \hbox to \hsize{\docketno\hfil- \the\pageno\ -\hfil\today}
}}

% initially set the values of \today and \docketno to blank
\def\today{}
\def\docketno{}


%%%%%%%%%%%%%%%%%%%%%%%%%%%%%%%%%%%%%%%%%%%%%%%%%%%%%%%%%%%%%%
% current date and docket number watermark macros
%%%%%%%%%%%%%%%%%%%%%%%%%%%%%%%%%%%%%%%%%%%%%%%%%%%%%%%%%%%%%%

% \timestamp use to place date and time information in the footline
\def\timestamp{
  \def\today{%
  \ifcase\month
  \or Jan\or Feb\or Mar%
  \or Apr\or May\or June%
  \or July\or Aug\or Sep%
  \or Oct\or Nov\or Dec%
  \fi
  \ \number\day, 
  \number\year\ 
  {%
    \count20=\time
    \count22=\count20
    \divide \count20 by 60
    \count21=\count20
    \multiply\count21 by -60
    \advance\count22 by \count21
    \ifnum\count20<10 0\fi
    \the\count20:%
    \ifnum\count22<10 0\fi
    \the\count22
  }%
}}

% \docketnumber{<docket-number>} places the docket number in the footline
\def\docketnumber#1{
  \makedef{docketno}{#1}
}


%%%%%%%%%%%%%%%%%%%%%%%%%%%%%%%%%%%%%%%%%%%%%%%%%%%%%%%%%%%%%%
% general macros
%%%%%%%%%%%%%%%%%%%%%%%%%%%%%%%%%%%%%%%%%%%%%%%%%%%%%%%%%%%%%%

\def\nl{\hfil\break}    % starts a new line
\def\ff{\vfill\eject}   % starts a new page


%%%%%%%%%%%%%%%%%%%%%%%%%%%%%%%%%%%%%%%%%%%%%%%%%%%%%%%%%%%%%%
% specification macros
%%%%%%%%%%%%%%%%%%%%%%%%%%%%%%%%%%%%%%%%%%%%%%%%%%%%%%%%%%%%%%

% section heading macro
% ---------------------
% 37 CFR 177 (c) The text of the specification sections [...]
% should be preceded by a section heading in uppercase and 
% without underlining or bold type. 
\def\heading#1{
  \everypar={\parnumbering}  % turns on paragraph numbering
  {
   \everypar={}
   \parindent 0pt
   \vskip 2\normalbaselineskip
   \baselineskip=\normalbaselineskip
   #1
   \par
}}

% contains the current paragraph number
\newcount\parno

% specification paragraph numbering macro
% --------------------------------------- 
% 37 CFR 1.52 (b) (6) Other than in a reissue application or
% reexamination proceeding, the paragraphs of the specification, other
% than in the claims or abstract, may be numbered at the time the
% application is filed, and should be individually and consecutively
% numbered using Arabic numerals, so as to unambiguously identify each
% paragraph. The number should consist of at least four numerals
% enclosed in square brackets, including leading zeros (e.g.,
% [0001]). The numbers and enclosing brackets should appear to the
% right of the left margin as the first item in each paragraph, before
% the first word of the paragraph, and should be highlighted in
% bold. A gap, equivalent to approximately four spaces, should follow
% the number. [...]
\def\parnumbering{%
  \advance\parno by 1
  \llap{\hbox to \parindent{\bf[%
    \ifnum\parno<1000 0\fi
    \ifnum\parno<100 0\fi
    \ifnum\parno<10 0\fi
    \the\parno
  ]\hss}}%
}


%%%%%%%%%%%%%%%%%%%%%%%%%%%%%%%%%%%%%%%%%%%%%%%%%%%%%%%%%%%%%%
% specification section headings (37 CFR 177 (b))
%%%%%%%%%%%%%%%%%%%%%%%%%%%%%%%%%%%%%%%%%%%%%%%%%%%%%%%%%%%%%%

% (1) Title of the invention [...] (unless included in the 
% application data sheet)
\def\title#1{
  \centered{#1}
}

% (1) Title of the invention which may be accompanied by an 
% introductory portion stating the name, citizenship, and residence 
% of the applicant (unless included in the application data sheet)
\def\inventor#1{
  \centered{
  by\nl#1
}}

\def\twoinventors#1#2{              % two inventors
  \centered{
  by\nl#1\nl and\nl#2
}}

\def\threeinventors#1#2#3{          % three inventors
  \centered{
  by\nl#1\nl#2\nl and\nl#3
}}

\def\fourinventors#1#2#3#4{         % four inventors
  \centered{
  by\nl#1\nl#2\nl#3\nl and\nl#4
}}

% (2) Cross-reference to related applications (unless included in the
% applicaiton data sheet)
\def\crossreference{
  \heading{
  CROSS-REFERENCE TO RELATED APPLICATIONS
}}

% (3) Statement regarding federally sponsored research or development
\def\federalresearch{
  \heading{
  STATEMENT REGARDING FEDERALLY SPONSORED
  RESEARCH OR DEVELOPMENT
}}

% (4) The names of the parties to a joint research agreement. 
\def\jointresearch{
  \heading{
  THE NAMES OF THE PARTIES TO A
  JOINT RESEARCH AGREEMENT
}}

% (5) Reference to a ``Sequence Listing,'' a table, or a computer
% program listing appendix submitted on a compact disc and an
% incorporation-by-reference of the material on the compact disc.
\def\compactdisc{
  \heading{
  INCORPORATION-BY-REFERENCE OF MATERIAL
  SUBMITTED ON A COMPACT DISC
}}

% (6) Background of the invention. 
\def\background{
  \heading{
  BACKGROUND OF THE INVENTION
}}

% (7) Brief summary of the invention. 
\def\summary{
  \heading{
  BRIEF SUMMARY OF THE INVENTION
}}

% (8) Brief description of the several views of the drawing. 
\def\drawings{
  \heading{
  BRIEF DESCRIPTION OF THE SEVERAL
  VIEWS OF THE DRAWINGS
}}

% (9) Detailed description of the invention. 
\def\description{
  \heading{
  DETAILED DESCRIPTION OF THE INVENTION
}}


%%%%%%%%%%%%%%%%%%%%%%%%%%%%%%%%%%%%%%%%%%%%%%%%%%%%%%%%%%%%%%
% 37 CFR 177 (b) (10) A claim or claims
%%%%%%%%%%%%%%%%%%%%%%%%%%%%%%%%%%%%%%%%%%%%%%%%%%%%%%%%%%%%%%

% begin claims
% ------------
% 37 CFR 1.52 (b) (3) The claim or claims must commence on a 
% separate physical sheet or electronic page.
\def\claims{
  \ff                           % start claims on a new page
  \heading{CLAIMS}  
  \heading{What Is Claimed Is:}  
  \everypar={\claimnumbering}   % each paragraph starts a claim
  \parindent 40pt
}

% use to indicate a new claim element
\def\el{
  \nl
  \hbox to \parindent{}
}     

% contains the current claim number
\newcount\claimno                     

% claim section paragraph numbering macro
\def\claimnumbering{%
  \advance\claimno by 1
  \hbox to 35pt{\the\claimno.\hss}
}


%%%%%%%%%%%%%%%%%%%%%%%%%%%%%%%%%%%%%%%%%%%%%%%%%%%%%%%%%%%%%%
% 37 CFR 177 (b) (11) Abstract of the disclosure. 
%%%%%%%%%%%%%%%%%%%%%%%%%%%%%%%%%%%%%%%%%%%%%%%%%%%%%%%%%%%%%%

% begin abstract
% --------------
% 37 CFR 1.52 (b) (4) The abstract must commence on a separate
% physical sheet or electronic page [...]
\def\abstract{     
  \ff                           % start abstract on a new page
  \heading{ABSTRACT OF THE DISCLOSURE}
  \everypar={}                  % end paragraph numbering
  \parindent 40pt
}


%%%%%%%%%%%%%%%%%%%%%%%%%%%%%%%%%%%%%%%%%%%%%%%%%%%%%%%%%%%%%%
% part referencing and claim labeling macros
%%%%%%%%%%%%%%%%%%%%%%%%%%%%%%%%%%%%%%%%%%%%%%%%%%%%%%%%%%%%%%

% macro for referencing a part name
% ---------------------------------
% eg. \ref{gogglecombination}{goggle combination 1} sets
% \googlecombination to print ``google combination 1''
\def\ref#1#2{%
  \makedef{#1}{#2}%
}

% macro for referencing a claim number
% ------------------------------------
% Creates a label for a claim which can be used in subsequent claims 
% to reference the former's claim number.  An example use is:
% ``\label{mainclaim} A mask and goggle device which comprises: ...
% The device of \mainclaim which ...''
\def\label#1{
  \count1=\claimno
  \advance\count1 by 1
  \makedef{#1}{claim \the\count1}
}


%%%%%%%%%%%%%%%%%%%%%%%%%%%%%%%%%%%%%%%%%%%%%%%%%%%%%%%%%%%%%%
% style macros
%%%%%%%%%%%%%%%%%%%%%%%%%%%%%%%%%%%%%%%%%%%%%%%%%%%%%%%%%%%%%%

% set paper size
% --------------
% 37 CFR 1.52 (a) (1) (ii) Either 21.0 cm by 29.7 cm (DIN size A4) or
% 21.6 cm by 27.9 cm (8 1/2 by 11 inches)
\pdfpagewidth 8.5in
\pdfpageheight 11in

% set font size
% -------------
% 37 CFR 1.52 (b) (2) (ii) Text written in a nonscript type font
% (e.g., Arial, Times Roman, or Courier, preferably a font size of 12)
% lettering style having capital letters which should be at least
% 0.3175 cm. (0.125 inch) high, but may be no smaller than 0.21 cm.
% (0.08 inch) high (e.g., a font size of 6);
\twelvepoints
\rm

% set line spacing
% ----------------
% 37 CFR 1.52 (b) (2) The specification (including the abstract and 
% claims) [...] must have: (i) Lines that are 1 1/2 or double spaced;
\baselineskip=2\normalbaselineskip            % 2 line spacing
\parskip=.35\baselineskip                     % paragraph vertical spacing
\vbadness=10000                               % avoid underfull page errors

% set margins
% -----------
% 37 CFR 1.52 (a) (1) (ii) [...] with each sheet including a top
% margin of at least 2.0 cm (3/4 inch), a left side margin of at least
% 2.5 cm (1 inch), a right side margin of at least 2.0 cm (3/4 inch),
% and a bottom margin of at least 2.0 cm (3/4 inch);
\hoffset=.25in                                % use a 1.25 inch margin 
\hsize=6in                                    % the text width is then 6 inches

% set ragged right output
% -----------------------
% 37 CFR 1.52 (b) (2) (iii) Only a single column of text.
\rightskip=0pt plus 30mm
\spaceskip=1\fontdimen2\font

% set pdf metadata
\pdfinfo
  { /Title (Patent Application) }

% set the indent for paragraph numbering
\parindent 14.5\fontdimen2\font

