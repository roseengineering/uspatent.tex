% -*- tex -*-
% uspatent.tex version 1.0 9/26/2012 gmagiros@gmail.com
%
% TeX macros for writing a United States Patent Application.
% Copyright (c) 2012 George Magiros
%
% This program is free software: you can redistribute it and/or modify
% it under the terms of the GNU General Public License as published by
% the Free Software Foundation, either version 3 of the License, or
% (at your option) any later version.
%
% This program is distributed in the hope that it will be useful,
% but WITHOUT ANY WARRANTY; without even the implied warranty of
% MERCHANTABILITY or FITNESS FOR A PARTICULAR PURPOSE.  See the
% GNU General Public License for more details.
%
% You should have received a copy of the GNU General Public License
% along with this program.  If not, see <http://www.gnu.org/licenses/>.

\input uspatent
\timestamp

\def\bs{$\backslash$}
\font\ss =cmss12                      % san serif font

\title{\TeX\ MACROS FOR WRITING A \nl UNITED STATES PATENT APPLICATION}
\inventor{George Magiros (United States)\nl gmagiros@gmail.com}

\heading{INTRODUCTION}

The file \ss uspatent.tex\rm\ provides \TeX\ macros for writing United
States utility patent applications.  Among its features are automatic
paragraph numbering, automatic claim numbering, docket number and time
watermarks, predefined section headings, and macros for referencing
parts and claim numbers.

To demonstrate how to write a patent application using the macros
described below, the file \ss example.tex\rm\ is provided.  To
generate the example application execute the command, \ss pdftex
example.tex\rm.  It is also highly recommended that you inspect the
USPTO Public PAIR data of the patent used in the example for
comparison purposes.  PAIR data is accessible from the USPTO's website
as well as from the patent's Google Patents search overview page.

\heading{WRITING YOUR PATENT APPLICATION}

First load the \TeX\ macros at the beginning of
your file.  Do this by typing:

\ss\bs input uspatent.tex
\rm

A current time watermark can be placed at the bottom of the page
by invoking the macro:

\ss\bs timestamp
\rm

A docket number also can be watermarked at the bottom of every page.
This is done with the macro:

\ss\bs docketnumber$\{${\it yourdocketnumber}$\}$
\rm

The first part of the patent application is the title.  It is created
with the macro:

\ss\bs title$\{${\it yourtitle}$\}$
\rm

You can include the names of the inventors with the title as well.
There are four macros to choose from depending on the number of
inventors.  Invoke one of the inventor macros after running the title
macro above.

          \ss\bs inventor$\{${\it inventor}$\}$\nl
\indent   \bs twoinventors$\{${\it firstinventor}$\}$$\{${\it secondinventor}$\}$\nl
\indent   \bs threeinventors$\{${\it firstinventor}$\}$$\{${\it secondinventor}$\}$$\{${\it thirdinventor}$\}$\nl
\indent   \bs fourinventors$\{${\it firstinventor}$\}$$\{${\it secondinventor}$\}$$\{${\it thirdinventor}$\}$\nl
\indent\indent $\{${\it fourthinventor}$\}$
          \rm

After the title comes the section headings.  The specification of the
patent application is divided into sections. The section heading macro is:

\ss\bs heading$\{${\it yoursectionheading}$\}$
\rm

As a convenience the following macros, each with their own predefined
section heading, are included (see MPEP \S608.01(a) for the text of
the predefined headings):

\indent   \ss\bs crossreference\nl
\indent   \bs federalresearch\nl
\indent   \bs jointresearch\nl
\indent   \bs compactdisc\nl
\indent   \bs background\nl
\indent   \bs summary\nl
\indent   \bs drawings\nl
\indent   \bs description
          \rm

Each paragraph in the specification, except in the claims and abstract
sections, is given a paragraph number.  This macro package will
automatically turn on paragraph numbering at the first section
heading.

In addition, the macro, \ss\bs nl\rm, can be used in the text of your sections to
force a new line.  The macro, \ss\bs ff\rm, is also available to force
a new page should, for example, a section heading cross pages.

\heading{THE CLAIMS AND ABSTRACT SECTION}

To start the claims section you invoke the claims macro:

\ss\bs claims
\rm

This macro turns off the previous paragraph numbering style of the
specification and replaces that style with claim numbering.  Claim
numbering counts each new paragraph as a new claim and numbers them
sequencially.  Claims also start on a new page so a page eject is
performed as well.

To set off each element of a claim and to prevent the start of a new
paragraph, which would upset the claim numbering macro, each claim
element should be prefixed with the macro:

\ss\bs el
\rm

The last section of the patent application is the abstract.  The abstract
is indicated with the macro:

\ss\bs abstract
\rm

The abstract macro forces a new page like the claims section. 
Paragraph numbering is not used in the abstract.

\heading{MACROS FOR REFERENCING PARTS AND CLAIMS}

Two additional macros are included to ease referencing part names and
claim numbers.  The first dynamically creates a macro to serve as a
shortcut for the name of the part.  This helps centralize part naming
and numbering in your \TeX\ file.  Instead of having to repeat each time
in the application the full name of the part and its part number, only
the macro need be typed.  See the \ss example.tex\rm{} file for
sample use.  The macro's format follows:

\indent   \bs ref$\{${\it macroname}$\}$$\{${\it fullnameofpart}$\}$\nl
\indent   \bs {\it macroname}$\{$$\}$
          \rm

The second macro is used in the claims section.  It also dynamically
creates a macro for later use.  Since claim numbers are automatically
generated keeping track of which claim number goes with which claim
can be difficult.  The purpose of this macro is to get \TeX\ to keep
track of the claim numbers for you.  When a labeled claim needs to be
referenced, such as within a dependant claim, you only need to type
the macro that was generated by that claim's label macro.  The
generated macro will output the word ``claim'' and the referenced
claim number, eg., ``claim 3'', for you.  Again see the sample
application in \ss example.tex\rm.  The macro's format is:

\indent   \bs label$\{${\it macroname}$\}$\nl
\indent   \bs {\it macroname}$\{$$\}$
          \rm


\heading{FILING YOUR PATENT APPLICATION}

These macros have not been used to actually draft and file,
electronically or otherwise, a patent application with the USPTO.
Please check the USPTO website, the relevant law, the USPTO's help
center, or other sources for the necessary filing and formatting
rules so your patent application does not get rejected.

Remember, all electronically filed applications must be run through
the USPTO's ``joboptions'' file to ensure compliance.  See the USPTO's
website for details.

\heading{LEGAL NOTICE}

{\it No guarantee or warranty, expressed or implied, is provided with
this macro package.  No attorney-client privilege is created through
the use, documentation, or continued support of this macro package.  No legal advice is
being offered or given.  All information provided is procedural and
does not constitute legal advice.  This is not attorney advertising.
See the license documents included in the package and the comments at
the beginning of each source file for more information.}

\indent Date: September 26, 2012\nl


\bye

